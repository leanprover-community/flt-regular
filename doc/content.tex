	\maketitle
\tableofcontents
\section{Introduction}
We prove Fermat's Last Theorem for regular primes and give some of the necessary background. It uses \cite{Samuel,marcus,washington}.

\section{Discriminants of number fields}
We recall basic facts about the discriminant.
\begin{lemma}
    \label{lemma:alt_definition_of_trace}
    \lean{trace_eq_sum_embeddings}
    \leanok
	Let $K$ be a number field, $\a \in K$ and let $\sigma_i$ be the embeddings of $K$ into $\CC$. Then \[\Tr_{K/\QQ}(\a) =\sum_i \sigma_i(\a) \].
\end{lemma}
\begin{proof}
  \leanok
  The proof is standard.
\end{proof}

\begin{lemma}\label{lemma:alt_definition_of_norm}
    \lean{algebra.norm_eq_prod_embeddings}
    \leanok
	Let $K$ be a number field, $\a \in K$ and let $\sigma_i$ be the embeddings of $K$ into $\CC$. Then \[N_{K/\QQ}(\a)=\prod_i \sigma_i(\a)  \].
\end{lemma}
\begin{proof}
  \leanok
  The proof is standard.
\end{proof}

\begin{definition}\label{defn_of_disc}
	\lean{algebra.discr}
	\leanok
	Let $A,K$ be commutative rings with $K$ and $A$-algebra. let $B=\{b_1,\dots,b_n\}$ be a set of elements in $K$. The discriminant of $B$ is defined as \[\Delta(B)=  \det \left (\begin{matrix} \Tr_{K/A}(b_1b_1) &\cdots& \Tr_{K/A}(b_1b_n)\\ \vdots & & \vdots \\  \Tr_{K/A}(b_nb_1) &\cdots& \Tr_{K/A}(b_nb_n)
	\end{matrix} \right ).\]
\end{definition}

\begin{lemma}\label{lem:lin_indep_iff_disc_ne_zero}
    \uses{defn_of_disc, lemma:alt_definition_of_norm}
    \leanok
    \lean{algebra.discr_not_zero_of_basis}
	Let $L/K$ be an extension of fields and let $B=\{b_1,\dots,b_n\}$ be a $K$-basis of $L$. Then $\Delta(B) \neq 0$.
\end{lemma}
\begin{proof}
  \leanok
  The proof is standard.
\end{proof}

\begin{lemma}\label{lem:disc_change_of_basis}
    \uses{defn_of_disc}
    \leanok
    \lean{algebra.discr_of_matrix_mul_vec}
	Let $K$ be a number field and $B,B'$ bases for $K/\QQ$. If $P$ denotes the change of basis matrix, then \[\Delta(B)=\det(P)^2 \Delta(B').\]
\end{lemma}
\begin{proof}
  \leanok
  The proof is standard.
\end{proof}

\begin{lemma}\label{lemma:disc_via_embs}
    \uses{defn_of_disc, trace_eq_sum_embeddings, lemma:alt_definition_of_trace}
    \leanok
    \lean{algebra.discr_eq_det_embeddings_matrix_reindex_pow_two}
	Let $K$ be a number field with basis $B=\{b_1,\dots,b_n\}$ and let $\sigma_1,\dots,\sigma_n$ be the embeddings of $K$ into $\CC$. Now let $M$ be the matrix  \[\left (\begin{matrix} \sigma_1(b_1) &\cdots& \sigma_1(b_n)\\ \vdots & & \vdots \\  \sigma_n(b_1) &\cdots& \sigma_n(b_n)
	\end{matrix} \right ).\] Then \[\Delta(B)=\det(M)^2.\]
\end{lemma}
\begin{proof}
    \leanok
	By Lemma \ref{lemma:alt_definition_of_trace} we know that  $\Tr_{K/\QQ}(b_i b_j)= \sum_k \sigma_k(b_i)\sigma_k(b_j)$ which is the same as the $(i,j)$ entry of $M^t M$. Therefore \[\det(T_B)=\det(M^t M)=\det(M)^2.\]
\end{proof}

\begin{lemma}\label{lemma:disc_of_prim_elt_basis}
	\lean{algebra.discr_power_basis_eq_prod}
	\leanok
    \uses{defn_of_disc,lemma:disc_via_embs}
	Let $K$ be a number field and $B=\{1,\a,\a^2,\dots,\a^{n-1}\}$ for some $\a \in K$. Then \[\Delta(B)=\prod_{i < j} (\sigma_i(\a)-\sigma_j(\a))^2\] where $\sigma_i$ are the embeddings of $K $ into $\CC$. Here $\Delta(B)$ denotes the discriminant.
\end{lemma}
\begin{proof}
    \leanok
	First we recall a classical linear algebra result relating to the Vandermonde matrix, which states that  \[\det \left (\begin{matrix} 1 & x_1& x_1^2&\cdots&x_1^{n-1} \\ \vdots & & & \vdots \\   1 & x_n& x_n^2&\cdots&x_n^{n-1}
	\end{matrix} \right ) =\prod_{i<j} (x_i-x_j).\] Combining this with Lemma \ref{lemma:disc_via_embs} gives the result.
\end{proof}

\begin{lemma}\label{lemma:diff_of_irr_pol}
    \lean{polynomial.aeval_root_derivative_of_splits}
    \leanok
	Let $f$ be a monic irreducible polynomial over a number field $K$ and let $\a$ be one of its roots in $\CC$. Then \[f'(\a)=\prod_{\beta \neq \a} (\a-\beta),  \] where the product is over the roots of $f$ different from $\a$.
\end{lemma}
\begin{proof}
    \leanok
	We first write $f(x)=(x-\a)g(x)$ which we can do (over $\CC$) as $\a$ is a root of $f$, where now $g(x)=\prod_{\beta \neq \a} (x-\beta)$. Differentiating we get \[f'(x)=g(x)+(x-\a)g'(x).\] If we now evaluate at $\a$ we get the result.
\end{proof}

\begin{lemma}\label{lemma:num_field_disc_in_terms_of_norm}
	\lean{algebra.of_power_basis_eq_norm}
	\leanok
	\uses{lemma:disc_of_prim_elt_basis,lemma:alt_definition_of_norm}
	Let $K=\QQ(\a)$ be a number field with $n=[K:\QQ]$ and let $B=\{1,\a,\a^2,\dots,\a^{n-1}\}$. Then \[\Delta(B)=(-1)^{\frac{n(n-1)}{2}}N_{K/\QQ}(m_\a'(\a))\] where $m_\a'$ is the derivative of $m_\a(x)$ (which we recall denotes the minimal polynomial of $\a$).
\end{lemma}
\begin{proof}
    \leanok
	By Lemma \ref{lemma:disc_of_prim_elt_basis} we have $\Delta(B)=\prod_{i < j}(\a_i-\a_j)^2$ where $\a_k:=\sigma_k(\a)$. Next, we note that the number of terms in this product is $1+2+\cdots+(n-1)=\frac{n(n-1)}{2}$. So if we write each term as $(\a_i-\a_j)^2=-(\a_i-\a_j)(\a_j-\a_i)$ we get \[\Delta(B)=(-1)^{\frac{n(n-1)}{2}}\prod_{i=1}^n \prod_{i \neq j} (\a_i-\a_j). \]

	Now, by lemmas \ref{lemma:diff_of_irr_pol} and \ref{lemma:alt_definition_of_norm} we see that \[N_{K/\QQ}(m_\a'(\a))=\prod_{i=1}^n m_\a'(\a_i)=\prod_{i=1}^n \prod_{i \neq j} (\a_i-\a_j),\] which gives the result.
\end{proof}

\begin{lemma}\label{lemma:norm_of_alg_int_is_int}
    \lean{algebra.is_integral_norm}
	\leanok
	If $K$ is a number field and $\a \in \OO_K$ then $N_{K/\QQ}(\a)$ is in $\ZZ$.
\end{lemma}
\begin{proof}
  \leanok
  The proof is standard.
\end{proof}

\begin{lemma}\label{lemma:trace_of_alg_int_is_int}
    \lean{algebra.is_integral_trace}
	\leanok
	If $K$ is a number field and $\a \in \OO_K$ then $\Tr_{K/\QQ}(\a)$ is in $\ZZ$.
\end{lemma}
\begin{proof}
  \leanok
  The proof is standard.
\end{proof}

\begin{lemma}\label{lemma:int_basis_int_disc}
    \lean{algebra.discr_is_integral}
    \leanok
    \uses{defn_of_disc,lemma:trace_of_alg_int_is_int}
	Let $K$ be a number field and $B=\{b_1,\dots,b_n\}$ be elements in $\OO_K$, then $\Delta(B) \in \ZZ$.
\end{lemma}
\begin{proof}
    \leanok
    Immediate by \ref{lemma:trace_of_alg_int_is_int}.
\end{proof}


\begin{lemma}\label{lemma:disc_int_basis}
	\lean{algebra.discr_mul_is_integral_mem_adjoin}
	\leanok
	\uses{lem:disc_change_of_basis,lemma:int_basis_int_disc}
	Let $K = \QQ(\alpha)$ be a number field, where $\alpha$ is an algebraic integer. Let $B = \{1, \alpha, \ldots, \alpha^{[K : \QQ] - 1} \}$ be the basis given by $\alpha$ and let $x \in \mathcal{O}_K$. Then $\Delta(B)x \in \ZZ[\alpha]$.
\end{lemma}

\begin{proof}
    \leanok
    See the Lean proof.
\end{proof}

\begin{lemma}\label{lemma:eis_crit_and_alg_ints}
	\lean{mem_adjoin_of_smul_prime_pow_smul_of_minpoly_is_eiseinstein_at}
	\leanok
	\uses{lemma:norm_of_alg_int_is_int,lemma:trace_of_alg_int_is_int}
	Let $K = \QQ(\alpha)$ be a number field, where $\alpha$ is an algebraic integer with minimal polynomial that is Eisenstein at $p$. Let $x \in \mathcal{O}_K$ such that $p^n x \in \ZZ[\alpha]$ for some $n$. Then $x \in \ZZ[\alpha]$.
\end{lemma}
\begin{proof}
    \leanok
    See the Lean proof.
\end{proof}

\section{Cyclotomic fields}

\begin{lemma}\label{lemma:cyclo_poly_deg}
    \lean{polynomial.degree_cyclotomic}
    \leanok
	For $n$ any integer, $\Phi_n$ (the $n$-th cyclotomic polynomial) is a polynomial of degree $\varphi(n)$ (where $\varphi$ is Euler's Totient function).
\end{lemma}
\begin{proof}
    \leanok
The proof is classical.
\end{proof}


\begin{lemma}\label{lemma:cyclo_poly_irr}
    \lean{polynomial.cyclotomic.irreducible}
    \leanok
	For $n$ any integer, $\Phi_n$ (the $n$-th cyclotomic polynomial) is an irreducible polynomial .
\end{lemma}
\begin{proof}
    \leanok
The proof is classical.
\end{proof}

\begin{lemma}\label{lem:discr_of_cyclo}
	\uses{lemma:num_field_disc_in_terms_of_norm,lemma:cyclo_poly_irr,lemma:cyclo_poly_deg}
	\leanok 
	\lean{is_cyclotomic_extension.rat.discr_prime_pow'}
	Let $\zeta_p$ be a $p$-th root of unity for $p$ an odd prime, let $\lam_p=1-\zeta_p$ and $K=\QQ(\zeta_p)$. Then \[\Delta(\{1,\zeta_p,\dots,\zeta_p^{p-2}\})=\Delta(\{1,\lam_p,\dots,\lam_p^{p-2}\})=(-1)^{\frac{(p-1)}{2}}p^{p-2}.\]
\end{lemma}
\begin{proof}
    \leanok
	First note $[K:\QQ]=p-1$.

	Since $\zeta_p=1-\lam_p$ we at once get $\ZZ[\zeta_p]=\ZZ[\lam_p]$ (just do double inclusion). Next, let $\a_i=\sigma_i(\zeta_p)$ denote the conjugates of $\zeta_p$, which is the same as the image of $\zeta_p$ under one of the embeddings $\sigma_i: \QQ(\zeta_p) \to \CC$. Now  by Proposition \ref{lemma:disc_of_prim_elt_basis} we have \begin{align*}\Delta(\{1,\zeta_p,\dots,\zeta_p^{p-2}\})=\prod_{i < j}  (\a_i-\a_j)^2 &=\prod_{i < j}  ((1-\a_i)-(1-\a_j))^2\\&=\Delta(\{1,\lam_p,\dots,\lam_p^{p-2}\})\end{align*}

	Now, by Proposition \ref{lemma:num_field_disc_in_terms_of_norm}, we have \[\Delta(\{1,\zeta_p,\cdots,\zeta_p^{p-2}\})=(-1)^{\frac{(p-1)(p-2)}{2}}N_{K/\QQ}(\Phi_p'(\zeta_p)  )\]
	Since $p$ is odd $(-1)^{\frac{(p-1)(p-2)}{2}}=(-1)^{\frac{(p-1)}{2}}$. Next, we see that \[\Phi_p'(x)=\frac{px^{p-1}(x-1)-(x^p-1)}{(x-1)^2}\] therefore \[\Phi_p'(\zeta_p)=-\frac{p\zeta_p^{p-1}}{\lam_p}.\]

	Lastly, note that $N_{K/\QQ}(\zeta_p)=1$, since this is the constant term in its minimal polynomial. Similarly, we see $N_{K/\QQ}(\lam_p)=p$. Putting this all together, we get \[N_{K/\QQ}(\Phi_p'(\zeta_p)  )=\frac{N_{K/\QQ}(p)N_{K\QQ}(\zeta_p)^{p-1}}{N_{K/\QQ}(-\lam_p)}=(-1)^{p-1}p^{p-2}=p^{p-2}\]
\end{proof}

\begin{theorem}\label{theorem:ring_of_ints_of_cyclo}
	\uses{lem:discr_of_cyclo,lemma:eis_crit_and_alg_ints,lemma:disc_int_basis}
	\leanok 
	\lean{is_cyclotomic_extension.rat.is_integral_closure_adjoing_singleton_of_prime}
	Let $\zeta_p$ be a $p$-th root of unity for $p$ an odd prime, let $\lam_p=1-\zeta_p$ and $K=\QQ(\zeta_p)$. Then $\OO_K=\ZZ[\zeta_p]=\ZZ[\lam_p]$.
\end{theorem}
\begin{proof}
    \leanok
    We need to prove is that $\OO_K=\ZZ[\zeta_p]$. The inclusion $\ZZ[\zeta_p] \subseteq \OO_K$ is obvious. Let now $x \in \OO_K$. By Lemma \ref{lemma:disc_int_basis} and Proposition \ref{lem:discr_of_cyclo}, there is $k \in \NN$ such that $p^k x \in \ZZ[\zeta_p]$. We conclude by Lemma \ref{lemma:eis_crit_and_alg_ints}.
\end{proof}

\begin{lemma}\label{lemma:alg_int_abs_val_one}\lean{mem_roots_of_unity_of_abs_eq_one}
	Let $\a$ be an algebraic integer all of whose conjugates have absolute value one. Then $\a$ is a root of unity.
\end{lemma}

\begin{lemma}\label{lem:roots_of_unity_in_cyclo}
	\lean{roots_of_unity_in_cyclo}
	Let $p$ be a prime, $K=\QQ(\zeta_p)$ $\a \in K$ such that there exists $n \in \NN$ such that $\a^n=1$, then $\a =\pm \zeta_p^k$ for some $k$.
\end{lemma}

\begin{proof}
	If $n$ is different to $p$ then $K$ contains a $2pn$-th root of unity. Therefore $\QQ(\zeta_{2pn}) \subset K$, but this cannot happen as $[K : \QQ]=p-1$ and $[\QQ(\zeta_{2pn}): \QQ ] = \varphi(2np)$.

\end{proof}

\begin{lemma}\label{lemma:unit_lemma}
	\lean{unit_lemma}
	\uses{lemma:alg_int_abs_val_one,lem:roots_of_unity_in_cyclo}
	Any unit $u$ in $\ZZ[\zeta_p]$ can be written in the form $\beta \zeta_p^k  $ with $k$ an integer and $\beta \in \RR$.
\end{lemma}

\begin{lemma}\label{lemma:fac_of_p_in_p_th_root}
	Let $p$ be a prime and $n=p^k$. Then \[p=u(1-\zeta_n)^{\varphi(n)}\] where $u \in \ZZ[\zeta_n]^{\times}$.
\end{lemma}

\begin{lemma}\label{lemma:ideals_mult_to_power}
    \lean{ideal.exists_eq_pow_of_mul_eq_pow}
    \leanok
	Let $R$ be a Dedekind domain, $p$ a prime and $\gotha,\gothb,\gothc$ ideals such that \[\gotha\gothb=\gothc^p\] and suppose $\gotha,\gothb$ are coprime. Then there exist ideals $\mathfrak{e},\mathfrak{d}$ such that \[\gotha=\mathfrak{e}^p \qquad \gothb=\mathfrak{d}^p \qquad \mathfrak{e}\mathfrak{d}=\gothc\]
\end{lemma}
\begin{proof}
 \leanok
 It follows from the unique decomposition of ideals in a Dedekind domain.
\end{proof}




\section{Fermats Last Theorem for regular primes}

\begin{lemma}\label{lem:flt_fact_2}
	\lean{flt_fact_2}
	Let $p \geq 5$ be an prime number, $\zeta_p$ a $p$-th root of unity and $x, y \in \ZZ$ coprime.

	 For $i \neq j$ we can write \[(\zeta_p^i-\zeta_p^j)=u(1-\zeta_p)\] with $u$ a unit in $\ZZ[\zeta_p]$. From this it follows that the ideals \[(x+y),(x+\zeta_py),(x+\zeta_p^2y),\dots,(x+\zeta_p^{p-1}y)\] are pairwise coprime.
\end{lemma}

\begin{proof}
	 Lemma \ref{lemma:fac_of_p_in_p_th_root} gives that $u$ is a unit. So all that needs to be proved is that the ideals are coprime. Assume not, then for some $i \neq j$ we have some prime ideal $\gothp$ dividing by $(x+y\zeta_p^i)$ and $(x+y\zeta_p^j)$. It must then also divide their sum and their difference, so we must have $\gothp | (1-\zeta_p)$ or $\gothp | y$. Similarly, $\gothp$ divides $\zeta_p^j(x+y\zeta_p^i)-\zeta_p^i(x+y\zeta_p^j)$ so $\gothp$ divides $x$ or $(1-\zeta_p)$. We can't have $\gothp$ dividing $x,y$ since they are coprime, therefore $\gothp |(1-\zeta_p)$. We know that since $(1-\zeta_p)$ has norm $p$ it must be a prime ideal, so $\gothp=(1-\zeta_p)$. Now, note that $x+y \equiv x+y\zeta_+p^i \equiv 0 \mod \gothp$. But since $x,y \in \ZZ$ this means we would have $x+y \equiv 0 \pmod p$, which implies $z^p \equiv 0 \pmod p$ which contradicts our assumptions.
\end{proof}

\begin{lemma}\label{lem:flt_fact_3}
	\lean{flt_fact_3}

	 	Let $p$ be an prime number, $\zeta_p$ a $p$-th root of unity and $\a \in \ZZ[\zeta_p]$. Then $\a^p$ is congruent to an integer modulo $p$.
\end{lemma}

\begin{proof}
	  Just use $(x+y)^p \equiv x^p + y^p \pmod p$ and that $\zeta_p$ is a $p$-th root of unity.
\end{proof}

\begin{lemma}\label{lem:flt_fact_4}
	\lean{lem:flt_fact_4}
	Let $p \geq 5$ be an prime number, $\zeta_p$ a $p$-th root of unity and $ \a \in \ZZ[\zeta_p]$. If there is an integer $n$ such that $\a/n \in \ZZ[\zeta_p]$, then $n$ divides each $a_i$.
\end{lemma}

\begin{proof}
	 Looking at $\a=a_0+a_1\zeta_p+\cdots+a_{p-1}\zeta_p^{p-1}$, if one of the $a_i$'s is zero and $\a/n \in \ZZ[\zeta_p]$, then $\a/n=\sum_i a_i/n \zeta_p^i$. Now, as $\a/n \in \ZZ[\zeta_p]$, pick the basis of $\ZZ[\zeta_p]$ which does not contain $\zeta_p$ (which is possible as any subset of $\{1,\zeta_p,\dots,\zeta_p^{p-1}\}$ with $p-1$ elements forms a basis of $\ZZ[\zeta_p]$.]). Then $\a=\sum_i b_i \zeta_p^i$ where $b_i \in \ZZ$. Therefore comparing coefficients, we get the result.
\end{proof}

\begin{lemma}\label{lem:flt_fact_5}
	\lean{lem:flt_fact_5}
	\uses{lem:flt_fact_4,lem:flt_fact_3}
	Let $p \geq 5$ be an prime number, $\zeta_p$ a $p$-th root of unity and $ \a \in \ZZ[\zeta_p]$.  Suppose that $x+y\zeta_p^i=u \a^p$ with $u \in \ZZ[\zeta_p]^\times$ and $\a \in \ZZ[\zeta_p]$. Then there is an integer $k$ such that \[x+y\zeta_p^i-\zeta_p^{2k}x-\zeta_p^{2k-i}y \equiv 0 \pmod p.\]
\end{lemma}

\begin{proof}
 Using lemma \ref{lemma:unit_lemma} we have $(x+y\zeta_p^i) =\beta \zeta_p^k \a^p$ which is equivalent to $\beta \zeta_p^k a \pmod p$ with $a$ and integer \ref{lem:flt_fact_3}). Now, if we consider the complex conjugate we have $\ov{(x+y\zeta_p^i)  }\equiv \beta \zeta_p^{-k}a \pmod p$. Looking at $(x+y\zeta_p^i)-\zeta_p^{2k}\ov{(x+y\zeta_p^i)  }$ then gives the result.


\end{proof}


\begin{lemma}\label{lemma:may_assume}
	\uses{lemma:unit_lemma,_lemma:fac_of_p_in_p_th_root,lemma:ideals_mult_to_power,theorem:ring_of_ints_of_cyclo}

	Let $p \geq 5$ be an prime number, $\zeta_p$ a $p$-th root of unity and $K=\QQ(\zeta_p)$.  Assume that we have $x,y,z \in \ZZ$ with $\gcd(xyz,p)=1$ and such that \[x^p+y^p=z^p.\]

Then without loss of generality, we may assume $x,y,z$ are pairwise coprime and \[x \not \equiv y \mod p.\]
\end{lemma}



\begin{proof}
The first part is easy.

Reducing modulo $p$, using Fermat's little theorem, you get that if $x \equiv y \equiv -z \pmod p$ then $3z \equiv 0 \pmod p$. But since $p >3$ this means $p |z$ but this contradicts $\gcd(xyz,p)=1$. Now, if $x \equiv y \pmod p$ then  $x \not \equiv -z \pmod p$ we can relabel $y,z$ so that wlog $x \not \equiv y$ (this uses that $p$ is odd).

\end{proof}

\begin{definition}\label{defn:is_regular_number}
	\lean{is_regular_number}
	\leanok
	A prime number $p$ is called regular if it does not divide the class number of $\QQ(\zeta_p)$.
\end{definition}


\begin{theorem}\label{theorem:FLT_case_one}
    \lean{flt_regular_case_one}
	\uses{defn:is_regular_number,lem:flt_fact_5,lem:flt_fact_4,lemma:may_assume,lem:flt_fact_2}
	Let $p$ be an odd regular prime. Then \[x^p+y^p=z^p\] has no solutions with $x,y,z \in \ZZ$ and $\gcd(xyz,p)=1$.
\end{theorem}
\begin{proof}
	First thing is to note that if $x^p+y^p=z^p$ then \[z^p=(x+y)(x+\zeta_py)\cdots(x+y\zeta_p^{p-1})\] as ideals. Then since by \ref{lem:flt_fact_2} we know the ideals are coprime, then by lemma \ref{lemma:ideals_mult_to_power} we have that each $(x+y\zeta_p^i)=\gotha^p$, for $\gotha$ some ideal. Note that, $[\gotha^p]=1$ in the class group. Now, since $p$ does not divide the size of the class group we have that $[\gotha]=1$ in the class group, so its principal. So we have $x+y\zeta_p^i=u_i\a_i^p$ with $u_i$ a unit. So by \ref{lem:flt_fact_5} we have some $k$ such that $x+y\zeta_p-\zeta_p^{2k}x-\zeta_p^{2k-1} \equiv 0 \pmod p$. If $1,\zeta_p,\zeta_p^{2k},\zeta_p^{2k-1}$ are distinct, then \ref{lem:flt_fact_4} (which uses that $p>3$) says that $p$  divides $x,y$, contrary to our assumption. So they cannot be distinct, but checking each case leads to a contradiction, therefore there cannot be any such solutions.
\end{proof}

\begin{theorem}\label{thm:Kummers_lemma}
	Let $p$ be a regular prime and let $u \in \ZZ[\zeta_p]^\times$. If $u^p \equiv a \mod p$ for some $a \in \ZZ$, then there exists $v \in \ZZ[\zeta_p]^\times$ such that $u=v^p$.
\end{theorem}	

\begin{theorem}\label{theorem:FLT_case_two}\lean{flt_regular_case_two}
	\lean{eq_pow_prime_of_unit_of_congruent}
	\uses{lemma:may_assume,defn:is_regular_number,thm:Kummers_lemma}
	Let $p$ be an odd regular prime. Then \[x^p+y^p=z^p\] has no solutions with $x,y,z \in \ZZ$ and $p | xyz$.
\end{theorem}


\begin{theorem}\label{FLT_regular}
	\lean{flt_regular}
	\uses{theorem:FLT_case_one,theorem:FLT_case_two}
	Let $p$ be an odd regular prime.  Then \[x^p+y^p=z^p\] has no solutions with $x,y,z \in \ZZ$ and $xyz \ne 0$.
\end{theorem}
