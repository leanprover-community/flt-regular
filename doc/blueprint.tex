\documentclass[11pt,a4paper]{amsart}

\title{Fermat's Last Theorem for regular primes}




%%==============================================================================
%
%              PACKAGES
%
%%==============================================================================


\usepackage[english]{babel}
\usepackage{enumerate}
\usepackage{setspace}
\usepackage[project=..]{blueprint}
\usepackage{amsfonts,amsthm,amsmath,amssymb,comment,url}

% Set up blueprint options


\home{https://leanprover-community.github.io/flt-regular/}
\github{https://github.com/leanprover-community/flt-regular/}


\usepackage[ocgcolorlinks]{hyperref}
\hypersetup{citecolor=blue,
	linkcolor=red}

%%==============================================================================
%
%              GEOMETRY
%
%%==============================================================================




%\addbibresource{bibliog.bib}
%\usepackage{fullpage}
%	\bibliography{bibliog}

\renewcommand{\baselinestretch}{1}




%%==============================================================================
%
%              ADDRESSES
%
%%==============================================================================

%\author{Christopher Birkbeck}	
%\address{Department of Mathematics, University College London, Gower street, London, WC1E 6BT}
%\email{c.birkbeck@ucl.ac.uk} 
%\urladdr{orcid.org/0000-0002-7546-9028}





%%==============================================================================
%
%              COMMENTS
%
%%==============================================================================



%%==============================================================================
%
%              THEOREM STYLES
%
%%==============================================================================
\theoremstyle{plain}
\newtheorem{theorem}[subsection]{Theorem}
\newtheorem*{theorem*}{Theorem}
\newtheorem{lemma}[subsection]{Lemma}
\newtheorem{cor}[subsection]{Corollary}
\newtheorem{con}[subsection]{Conjecture}
\newtheorem*{con*}{Conjecture}
\newtheorem{prop}[subsection]{Proposition}
\newtheorem*{prop*}{Proposition}
\theoremstyle{definition}
\newtheorem{definition}[subsection]{Definition}
\newtheorem{exmp}[subsection]{Example}
\newtheorem{nexmp}[subsection]{Non-example}
\theoremstyle{definition}
\newtheorem{rmrk}[subsection]{Remark}
\newtheorem*{rmrk*}{Remark}
\newtheorem{warn}[subsection]{Warning}
\newtheorem*{warn*}{Warning}
\newtheorem{exmps}[subsection]{Examples} 
\newtheorem*{exmp*}{Example}
\newtheorem{obs}[subsection]{Observation}
\newtheorem*{obs*}{Observation}
\newtheorem{nota}[subsection]{Notation}
\newtheorem{question}[subsection]{Exercise}
\newtheorem{num}[subsection]{\unskip}
\newtheorem{alg}[subsection]{Algorithm}




%%==============================================================================
%
%              MACROS
%
%%==============================================================================

%Shortcuts: mathbb
\newcommand{\Q}{\mathbb{Q}}
\renewcommand{\O}{\mathcal{O}}
\newcommand{\Tr}{\operatorname{Tr}}
\newcommand{\Ql}{\mathbb{Q}_\ell}
\newcommand{\Qp}{\mathbb{Q}_p}
\renewcommand{\c}{\mathfrak c}

\newcommand{\Z}{\mathbb{Z}}
\newcommand{\ZZ}{\mathbb{Z}}
\newcommand{\Zp}{\mathbb{Z}_p}
\newcommand{\Zpthick}{\ZZ_p^\times}

\newcommand{\R}{\mathbb{R}}
\newcommand{\RR}{\mathbb{R}}

\newcommand{\A}{\mathbb{A}}
\newcommand{\D}{\mathbb{D}}
\newcommand{\C}{\mathbb{C}}
\newcommand{\Cp}{\mathbb{C}_p}
\newcommand{\N}{\mathbb{N}}
\newcommand{\F}{\mathbb{F}}

\newcommand{\TT}{\mathbb{T}}
\DeclareMathOperator{\bA}{\mathbb{A}}
\DeclareMathOperator{\bG}{\mathbb{G}}
\DeclareMathOperator{\bH}{\mathbb{H}}
\DeclareMathOperator{\bT}{\mathbb{T}}
\DeclareMathOperator{\bV}{\mathbb{V}}
\DeclareMathOperator{\bW}{\mathbb{W}}
\newcommand{\bD}{\mathbb{D}}
\def\AA{\mathbb{A}}
\def\CC{\mathbb{C}}
\def\FF{\mathbb{F}}
\def\GG{\mathbb{G}}
\def\NN{\mathbb{N}}
\renewcommand{\P}{\mathbb{P}}
\renewcommand{\A}{\mathbb{A}}
\newcommand{\G}{\mathbb{G}}
\def\PP{\mathbb{P}}
\def\QQ{\mathbb{Q}}
\def\TT{\mathbb{T}}
\def \Nm {N_{K/\QQ}}


% Shortcuts: mathcal
\newcommand{\f}{\mathcal{F}}
\newcommand{\T}{\mathcal{T}}
\newcommand{\uhp}{\mathcal{H}}
\newcommand{\roi}{\mathcal{O}}
\newcommand{\DD}{\mathcal{D}}
\newcommand{\cA}{\mathcal{A}}
\newcommand{\cU}{\mathcal{U}}
\newcommand{\cB}{\mathcal{B}}
\newcommand{\cC}{\mathcal{C}}
\newcommand{\cD}{\mathcal{D}}
\newcommand{\cE}{\mathcal{E}}
\newcommand{\cF}{\mathcal{F}}
\newcommand{\cH}{\mathcal{H}}
\newcommand{\cM}{\mathcal{M}}
\newcommand{\cP}{\mathcal{P}}
\newcommand{\cR}{\mathcal{R}}
\newcommand{\cV}{\mathcal{V}}
\newcommand{\cW}{\mathcal{W}}
\newcommand{\cX}{\mathcal{X}}
\newcommand{\cO}{\mathcal{O}}
\newcommand{\cG}{\mathcal{G}}
\newcommand{\cY}{\mathcal{Y}}
\newcommand{\cI}{\mathcal{I}}
\newcommand{\cZ}{\mathcal{Z}}
\newcommand{\UU}{\mathcal{U}}
\newcommand{\W}{\mathcal{W}}
\def \D {\mathcal{D}}
\def \E {\mathcal{E}}
\def \B {\mathcal{B}}
\def \M {\mathcal{M}}
\def \Y {\mathcal{Y}}
\def \XX {\mathcal{X}}
\def \XXt {\mathcal{X}^{\ast}}
\newcommand*\pp{{\rlap{\('\)}}}
\newcommand*\mA{\mathcal A} 
\newcommand*\mB{\mathcal B} 
\newcommand*\mC{\mathcal C}

% Shortcuts: mathscr


\newcommand{\fc}{\mathfrak{c}}
\newcommand{\ff}{\mathfrak{f}}
\newcommand{\fg}{\mathfrak{g}}
\newcommand{\fH}{\mathfrak{H}}
\newcommand{\fn}{\mathfrak{n}}
\newcommand{\fm}{\mathfrak{m}}
\newcommand{\m}{\mathfrak{m}}
\newcommand{\n}{N}
\newcommand{\fp}{\mathfrak{p}}
\newcommand{\pri}{\mathfrak{p}}
\newcommand{\fW}{\mathfrak{W}}
\newcommand{\fX}{\mathfrak{X}}
\def\gothb{\mathfrak{b}}
\def\gothc{\mathfrak{c}}
\def\ee{\mathfrak{e}}
\def\gothm{\mathfrak{m}}
\def\gothd{\mathfrak{d}}
\def\gotho{\mathfrak{o}}
\def\goths{\mathfrak{s}}
\def\gothz{\mathfrak{z}}
\def \gothP{\mathfrak{P}}
\def \gothQ{\mathfrak{Q}}
\def \nn {N}
\def \zz {\mathfrak{z}}
\def \MM {\mathfrak{M}}
\def\gothp{\mathfrak{p}}
\def\ps{\mathfrak{p}}
\def\gothq{\mathfrak{q}}




\def \a{\alpha}
\def \e {\epsilon}
\def \aa {\pmb{a}}
\def \g {\gamma}
\def \GA {\Gamma}
\def \w {\omega}
\def \LL {\mathcal{L}}
\def \W {\mathcal{W}}
\def \p {\partial}
\def \ll {\lambda}
\def \L {\Lambda}
\def \st {\star}
\def \bu {\bullet}

\def \s {\sigma}
\def \Si {\Sigma}
\def \lam {\lambda}
\def \ii {\iota}
\def \d {\mathfrak{d}}
\def \OO {\mathcal{O}}
\def \OF {\mathcal{O}_F}
\def \OFU {\mathcal{O}_F^{\times,+}}
\def \OOO {\mathscr{O}}

\def \CD {\mathfrak{D}}
\def \CX {\mathscr{X}}
\def \CZ {\mathscr{Z}}
\def \CM {\mathscr{M}}
\def \Z {\ZZ}
\def \s {\sigma}
\def \SS {S_{k}}
\def \PGI {\mathrm{P}\Gamma_{\!\scaleto{0}{3pt}}(p^n)}
\def \GI {\Gamma_{\!\scaleto{0}{3pt}}(p^n)}
\def \GIst {\Gamma_{\!\scaleto{0}{3pt}}^*(p^n)}
\def \un {\underline}
\def \ov{\overline}
\def \wt{\widetilde}
\def \wh{\widehat}
\def \H{\widehat{H}}
\def \HHH {\mathcal{H}}
\def \U {\mathcal{U}}


\def\gotha{\mathfrak{a}}
\def \tr {\ttr_{F/\QQ}}
\def \eU {\e_\k}


\def \k {\kappa}
\def \Norm {N_{F/\QQ}}


% Shortcuts: Operators
\DeclareMathOperator{\coker}{coker}
\DeclareMathOperator{\Sch}{Sch}
\DeclareMathOperator{\Set}{Set}
\DeclareMathOperator{\hH}{H}
\DeclareMathOperator{\tor}{tor}
\DeclareMathOperator{\et}{et}
\DeclareMathOperator{\qproet}{qproet}
\DeclareMathOperator{\End}{End}
\DeclareMathOperator{\Lie}{Lie}
\DeclareMathOperator{\Sym}{Sym}
\DeclareMathOperator{\Ker}{Ker}
\DeclareMathOperator{\Sp}{Sp}
\DeclareMathOperator{\Ext}{Ext}
\DeclareMathOperator{\val}{val}
\DeclareMathOperator{\im}{Im}
\DeclareMathOperator{\lcm}{lcm}

\DeclareMathOperator{\unn}{univ}
\DeclareMathOperator{\Spa}{Spa}
\DeclareMathOperator{\Spf}{Spf}
\DeclareMathOperator{\Spec}{Spec}
\DeclareMathOperator{\NP}{NP}
\DeclareMathOperator{\Inv}{Inv}
\newcommand{\an}{\mathrm{an}}
\newcommand{\ad}{\mathrm{ad}}
\newcommand{\Ig}{\mathrm{Ig}}
\DeclareMathOperator{\ab}{ab}
\DeclareMathOperator{\fs}{fs}
\newcommand{\id}{\operatorname{id}}
\DeclareMathOperator{\Stab}{Stab}
\DeclareMathOperator{\coinf}{Coinf}
\DeclareMathOperator{\trr}{tr}
\DeclareMathOperator{\Det}{Det}
\DeclareMathOperator{\nrd}{nrd}
\DeclareMathOperator{\ttr}{Tr}
\DeclareMathOperator{\suup}{sup}
\DeclareMathOperator{\Tor}{Tor}

\newcommand{\AIP}{\mathrm{AIP}}
\DeclareMathOperator{\Fred}{Fred}
\DeclareMathOperator{\Frob}{Frob}
\DeclareMathOperator{\Hdg}{Hdg}
\DeclareMathOperator{\Coeq}{Coeq}
\DeclareMathOperator{\Cor}{Cor}
\newcommand{\cyc}{{\text{cyc}}}
\newcommand{\cts}{{\text{cts}}}
\DeclareMathOperator{\disc}{Disc}
\DeclareMathOperator{\Gal}{Gal}
\DeclareMathOperator{\SL}{SL}
\DeclareMathOperator{\Sh}{Sh}
\DeclareMathOperator{\Gr}{Gr}
\DeclareMathOperator{\Vol}{Vol}
\DeclareMathOperator{\GL}{GL}
\DeclareMathOperator{\GSp}{GSp}
\DeclareMathOperator{\Isom}{Isom}
\DeclareMathOperator{\Hom}{Hom}
\DeclareMathOperator{\Ind}{Ind}
\DeclareMathOperator{\inv}{inv}
\DeclareMathOperator{\Res}{Res}
\DeclareMathOperator{\smcy}{smcy}
\DeclareMathOperator{\Trace}{Trace}
\DeclareMathOperator{\Cl}{Cl}
\DeclareMathOperator{\Inf}{Inf}
\DeclareMathOperator{\Aut}{Aut}
\DeclareMathOperator{\Mapc}{\operatorname{Map}_{\cts}}
\newcommand{\HTf}{\HT_{\mathrm{fin}}}
\newcommand{\hcg}{\mathrm{H}_{\mathrm{c}}^g}


%
%
%              END PREAMBLE
%
%%==============================================================================
\begin{document}
 	\maketitle
\tableofcontents
\section{Introduction}
We prove Fermat's Last Theorem for regular primes and give some of the necessary background. It uses \cite{Samuel,marcus,washington}.

\section{Discriminants of number fields}
\begin{lemma}\label{lemma:alt_definition_of_norm_and_trace}
	Let $K$ be a number field, $\a \in K$ and let $\sigma_i$ be the embeddings of $K$ into $\CC$. Then \[\Tr_{K/\QQ}(\a) =\sum_i \sigma_i(\a) \qquad N_{K/\QQ}(\a)=\prod_i \sigma_i(\a)  \]
\end{lemma}

\begin{definition}\label{defn_of_disc}
	\lean{algebra.discriminant}
	Let $A,K$ be commutative rings with $K$ and $A$-algebra. let $B=\{b_1,\dots,b_n\}$ be a set of elements in $K$. The discriminant of $B$ is defined as \[\Delta(B)=  \det \left (\begin{matrix} \Tr_{K/A}(b_1b_1) &\cdots& \Tr_{K/A}(b_1b_n)\\ \vdots & & \vdots \\  \Tr_{K/A}(b_nb_1) &\cdots& \Tr_{K/A}(b_nb_n)
	\end{matrix} \right ).\]
\end{definition}

\begin{lemma}\label{lem:lin_indep_iff_disc_ne_zero}
	Let $K$ be a number field and let $B=\{b_1,\dots,b_n\}$ be a set of elements in $K$. Then $\Delta(B) \neq 0$ if and only if the elements in $B$ are linearly independent.
\end{lemma}

\begin{lemma}\label{lem:disc_change_of_basis}
	Let $K$ be a number field and $B,B'$ bases for $K/\QQ$. If $P$ denotes the change of basis matrix, then \[\Delta(B)=\det(P)^2 \Delta(B').\]
\end{lemma}

\begin{lemma}\label{lemma:disc_via_embs}
	Let $K$ be a number field with basis $B=\{b_1,\dots,b_n\}$ and let $\sigma_1,\dots,\sigma_n$ be the embeddings of $K$ into $\CC$. Now let $M$ be the matrix  \[\left (\begin{matrix} \sigma_1(b_1) &\cdots& \sigma_1(b_n)\\ \vdots & & \vdots \\  \sigma_n(b_1) &\cdots& \sigma_n(b_n)
	\end{matrix} \right ).\] Then \[\Delta(B)=\det(M)^2.\]
\end{lemma}


\begin{proof}
	By Proposition \ref{lemma:alt_definition_of_norm_and_trace} we know that  $\Tr_{K/\QQ}(b_i b_j)= \sum_k \sigma_k(b_i)\sigma_k(b_j)$ which is the same as the $(i,j)$ entry of $M^t M$. Therefore \[\det(T_B)=\det(M^t M)=\det(M)^2.\]
\end{proof}

\begin{lemma}\label{lemma:disc_of_prim_elt_basis}
	\uses{lemma:disc_via_embs}
	Let $K$ be a number field and $B=\{1,\a,\a^2,\dots,\a^{n-1}\}$ for some $\a \in K$. Then \[\Delta(B)=\prod_{i < j} (\sigma_i(\a)-\sigma_j(\a))^2\] where $\sigma_i$ are the embeddings of $K $ into $\CC$. Here $\Delta(B)$ denotes the discriminant.
\end{lemma}
\begin{proof}
	First we recall a classical linear algebra result relating to the Vandermonde matrix, which states that  \[\det \left (\begin{matrix} 1 & x_1& x_1^2&\cdots&x_1^{n-1} \\ \vdots & & & \vdots \\   1 & x_n& x_n^2&\cdots&x_n^{n-1}
	\end{matrix} \right ) =\prod_{i<j} (x_i-x_j).\] Combining this with Proposition \ref{lemma:disc_via_embs} gives the result.
\end{proof}

\begin{lemma}\label{lemma:diff_of_irr_pol}
	Let $f$ be a monic irreducible polynomial over a number field $K$ and let $\a$ be one of its roots in $\CC$. Then \[f'(\a)=\prod_{\beta \neq \a} (\a-\beta),  \] where the product is over the roots of $f$ different from $\a$.
\end{lemma}

\begin{proof}
	We first write $f(x)=(x-\a)g(x)$ which we can do (over $\CC$) as $\a$ is a root of $f$, where now $g(x)=\prod_{\beta \neq \a} (x-\beta)$. Differentiating we get \[f'(x)=g(x)+(x-\a)g'(x).\] If we now evaluate at $\a$ we get the result.
\end{proof}

\begin{lemma}\label{lemma:num_field_disc_in_terms_of_norm}
	\uses{lemma:disc_of_prim_elt_basis,lemma:alt_definition_of_norm_and_trace}
	Let $K=\QQ(\a)$ be a number field with $n=[K:\QQ]$ and let $B=\{1,\a,\a^2,\dots,\a^{n-1}\}$. Then \[\Delta(B)=(-1)^{\frac{n(n-1)}{2}}N_{K/\QQ}(m_\a'(\a))\] where $m_\a'$ is the derivative of $m_\a(x)$ (which we recall denotes the minimal polynomial of $\a$).
\end{lemma}
\begin{proof}
	By Proposition \ref{lemma:disc_of_prim_elt_basis} we have $\Delta(B)=\prod_{i < j}(\a_i-\a_j)^2$ where $\a_k:=\sigma_k(\a)$. Next, we note that the number of terms in this product is $1+2+\cdots+(n-1)=\frac{n(n-1)}{2}$. So if we write each term as $(\a_i-\a_j)^2=-(\a_i-\a_j)(\a_j-\a_i)$ we get \[\Delta(B)=(-1)^{\frac{n(n-1)}{2}}\prod_{i=1}^n \prod_{i \neq j} (\a_i-\a_j). \]

	Now, by lemma \ref{lemma:diff_of_irr_pol} and Proposition \ref{lemma:alt_definition_of_norm_and_trace} we see that \[N_{K/\QQ}(m_\a'(\a))=\prod_{i=1}^n m_\a'(\a_i)=\prod_{i=1}^n \prod_{i \neq j} (\a_i-\a_j),\] which gives the result.
\end{proof}

\begin{lemma}\label{lemma:disc_change_of_basis}
	Let $K$ be a number field and $B,B'$ bases for $K/\QQ$. If $P$ denotes the change of basis matrix, then \[\Delta(B)=\det(P)^2 \Delta(B').\]
\end{lemma}

\begin{lemma}\label{lemma:norm_trace_of_alg_int_is_int}
	If $K$ is a number field and $\a \in \OO_K$ then $\Tr_{K/\QQ}(\a)$ and $N_{K/\QQ}(\a)$ are both in $\ZZ$.
\end{lemma}

\begin{lemma}\label{lemma:int_basis_int_disc}
	Let $K$ be a number field and $B=\{b_1,\dots,b_n\}$ be elements in $\OO_K$, then $\Delta(B) \in \ZZ$.
\end{lemma}

\begin{lemma}\label{lemma:disc_int_basis}
	\uses{lemma:disc_change_of_basis,lemma:int_basis_int_disc}
	Let $K$ be a number field and $B=\{b_1,\dots,b_n\}$ be a basis for $K/\QQ$ consisting of algebraic integers. If $B$ is not an integral basis then there exists an algebraic integer of the form \[\a=\frac{x_1b_1+\cdots+x_nb_n}{p}\] where $p$ is a prime and $x_i \in \{0,\dots,p-1\}$ with not all $x_i$ zero. Moreover, if $x_i \neq 0$ and we let $B'$ be the basis obtained by replacing $b_i$ with $\a$, then \[\Delta(B')= \frac{x_i^2}{p^2} \Delta(B).\] In particular $p^2 \mid \Delta(B)$.
\end{lemma}

\begin{proof}
	If $B$ is not an integral basis then we can find some element $\phi \in \OO_K$ such that \[\phi=y_1b_1+\dots y_nb_n\] with not all the $y_i$ in $\ZZ$. So, let $N$ be the least common multiple of the denominators of the $y_i$ (meaning $Ny_i \in \ZZ$ for all $i$). Now, let $p$ be a prime factor of $N$. If we now consider $(N/p)\phi$ then all of the coefficients of $b_i$ are in $\frac{1}{p} \ZZ$ (so they have denominator $1$ or $p$.) and at least one of them has denominator $p$ (since not all the $y_i$ where in $\ZZ$). So by relabelling, wlog we can assume \[\phi=y_1b_1+\dots y_nb_n\] with $y_i \in  \frac{1}{p} \ZZ$

	Now look at \[\psi:=\lfloor y_1 \rfloor b_1+\cdots+\lfloor y_n \rfloor b_n\] (here $\lfloor x \rfloor$ denotes the integer part of $x$). The both $\psi$ and $\phi$ are algebraic integers (as the $b_i$ are algebraic integers). Therefore, so is $\theta=\phi-\psi$. By construction, $\theta$ has coefficients of the for $\frac{x_i}{p}:=y_i-\lfloor y_i \rfloor$ where $x_i \in \{0,\dots,p-1\}$ and not all the $x_i$ are zero (since, again, not all the $y_i$ were in $\ZZ$). This gives the first part of the lemma.

	Now, assume $x_i \neq 0$, then let us replace $b_i \in B$ with $\theta$ to get a new basis $B'$ which again consists of algebraic integers. Next, we note that the change of basis matrix from $B$ to $B'$ is

	\[ \left( \begin{matrix}
		1&0&\cdots& \frac{x_1}{p}& \cdots &0 \\
		0&1&\cdots& \frac{x_2}{p}& \cdots &0 \\
		\vdots& \vdots&&\vdots&&\vdots \\
		0&0&\cdots& \frac{x_n}{p}& \cdots &1 \\
	\end{matrix}                          \right) \](here the column of $x_j/p$'s is in the $i$-th column).

	This matrix has determinant $\frac{x_i}{p}.$ Therefore, by 	Proposition \ref{lemma:disc_change_of_basis} we see that $\Delta(B')=\frac{x_i^2}{p^2}\Delta(B)$. But both $\Delta(B),\Delta(B')$ are  in $\ZZ$ by Proposition \ref{lemma:int_basis_int_disc}, therefore $p^2 \mid \Delta(B)$ giving the result.
\end{proof}

\begin{lemma}\label{lemma:eis_crit_and_alg_ints}
	\uses{lemma:norm_trace_of_alg_int_is_int}
	Let $K=\QQ(\a)$ and $\a$ be an algebraic integer such that $m_\a$ satisfies Eisensteins Criterion for a prime $p$. Then none of the elements \[\phi=\frac{1}{p}(x_0+x_1\a+\cdots+x_{n-1}\a^{n-1})\] is an algebraic integer, where $n=\deg(m_\a)$ and $x_i \in \{0,\dots,p-1\}.$
\end{lemma}


\begin{proof}
	We will only prove the case when $m_\a$ is Eisenstein, since the proof of the more general case is identical.

	Suppose for contradiction that $\phi \in \OO_K$ and let $x_d$ be the first non-zero coefficient, so \[\phi=\frac{1}{p}(x_d\a^d+x_{d+1}\a^{d+1}+\cdots+x_{n-1}\a^{n-1}) \in \OO_K.\]

	Now, rewrite this as $\phi=\frac{1}{p}(x_d\a^d +\a^{d+1}\beta)$ for some $\beta \in \OO_K$. Next, multiply through by $\a^{n-1-d}$, then we have \[\frac{x_d\a^{n-1}}{p}+\frac{\a^n\beta}{p} \in \OO_K.\]

	Now, since  $m_\a$ satisfies Eisenstein at $p$, we see that $\a^n=pf(\a)$ for some $f \in \ZZ[x]$ and therefore the above gives us that \[\frac{x_d\a^{n-1}}{p}+\beta f(\a) \in \OO_K.\] and thus \[\frac{x_d\a^{n-1}}{p} \in \OO_K.\]

	Lets now calculate the norm of this: \[N_{K/\QQ} \left(\frac{x_d\a^{n-1}}{p} \right)=\frac{x_d^n N_{K/\QQ}(\a)^{n-1}}{p^n}.\] By Eisenstein the constant coefficient of $m_\a$ is divisible by $p$ but not $p^2$, so since the constant coefficient of $m_\a$ is $N_{K/\QQ}(\a)$ we see that $N_{K/\QQ}(\a)=p a$ where $p \nmid a$. Therefore we have \[N_{K/\QQ} \left(\frac{x_d\a^{n-1}}{p} \right)=\frac{x_d^n p^{n-1}a^{n-1}}{p^n}= \frac{x_d^n a^{n-1}}{p}.\] But this cant be in $\ZZ$ since $p$ doesn't divide $x_d$ or $a$, and this gives us a contradiction since  Proposition \ref{lemma:norm_trace_of_alg_int_is_int} says that the norm of an algebraic integer must be an integer. So $\phi$ couldn't have been an algebraic integer.


\end{proof}

\section{Cyclotomic fields}


\begin{lemma}\label{lemma:cyclo_poly}\lean{polynomial.degree_cyclotomic}
	For $n$ any integer, $\Phi_n$ (the $n$-th cyclotomic polynomial) is an irreducible polynomial of degree $\varphi(n)$ (where $\varphi$ is Euler's Totient function).
\end{lemma}




\begin{theorem}\label{theorem:ring_of_ints_of_cyclo}
	\uses{lemma:num_field_disc_in_terms_of_norm,lemma:disc_int_basis,lemma:eis_crit_and_alg_ints}
	Let $\zeta_p$ be a $p$-th root of unity for $p$ an odd prime, let $\lam_p=1-\zeta_p$ and $K=\QQ(\zeta_p)$. Then $\OO_K=\ZZ[\zeta_p]=\ZZ[\lam_p]$ moreover \[\Delta(\{1,\zeta_p,\dots,\zeta_p^{p-2}\})=\Delta(\{1,\lam_p,\dots,\lam_p^{p-2}\})=(-1)^{\frac{(p-1)}{2}}p^{p-2}.\]
\end{theorem}

\begin{proof}
	First note $[K:\QQ]=p-1$.

	Since $\zeta_p=1-\lam_p$ we at once get $\ZZ[\zeta_p]=\ZZ[\lam_p]$ (just do double inclusion). Next, let $\a_i=\sigma_i(\zeta_p)$ denote the conjugates of $\zeta_p$, which is the same as the image of $\zeta_p$ under one of the embeddings $\sigma_i: \QQ(\zeta_p) \to \CC$. Now  by Proposition \ref{lemma:disc_of_prim_elt_basis} we have \begin{align*}\Delta(\{1,\zeta_p,\dots,\zeta_p^{p-2}\})=\prod_{i < j}  (\a_i-\a_j)^2 &=\prod_{i < j}  ((1-\a_i)-(1-\a_j))^2\\&=\Delta(\{1,\lam_p,\dots,\lam_p^{p-2}\})\end{align*}

	Now, by Proposition \ref{lemma:num_field_disc_in_terms_of_norm}, we have \[\Delta(\{1,\zeta_p,\cdots,\zeta_p^{p-2}\})=(-1)^{\frac{(p-1)(p-2)}{2}}N_{K/\QQ}(\Phi_p'(\zeta_p)  )\]
	Since $p$ is odd $(-1)^{\frac{(p-1)(p-2)}{2}}=(-1)^{\frac{(p-1)}{2}}$. Next, we see that \[\Phi_p'(x)=\frac{px^{p-1}(x-1)-(x^p-1)}{(x-1)^2}\] therefore \[\Phi_p'(\zeta_p)=-\frac{p\zeta_p^{p-1}}{\lam_p}.\]

	Lastly, note that $N_{K/\QQ}(\zeta_p)=1$, since this is the constant term in its minimal polynomial. Similarly, we see $N_{K/\QQ}(\lam_p)=p$. Putting this all together, we get \[N_{K/\QQ}(\Phi_p'(\zeta_p)  )=\frac{N_{K/\QQ}(p)N_{K\QQ}(\zeta_p)^{p-1}}{N_{K/\QQ}(-\lam_p)}=(-1)^{p-1}p^{p-2}=p^{p-2}\]

	So the last thing we need to prove is that $\OO_K=\ZZ[\zeta_p]$. From the calculation we just did, the only prime dividing the discriminant is $p$, therefore Lemma \ref{lemma:disc_int_basis} tells us the only prime we need to check is $p$. But from Lemma \ref{lemma:eis_crit_and_alg_ints} we know that dividing by $p$ wont give us any new integral elements, so this must be an integral basis which give the result.
\end{proof}

\begin{lemma}\label{lemma:alg_int_abs_val_one}\lean{mem_roots_of_unity_of_abs_eq_one}
	Let $\a$ be an algebraic integer all of whose conjugates have absolute value one. Then $\a$ is a root of unity.
\end{lemma}


\begin{lemma}\label{lemma:unit_lemma}
	\lean{unit_lemma}
	\uses{lemma:alg_int_abs_val_one}
	Any unit $u$ in $\ZZ[\zeta_p]$ can be written in the form $\beta \zeta_p^k  $ with $k$ an integer and $\beta \in \RR$.
\end{lemma}

\begin{lemma}\label{lemma:fac_of_p_in_p_th_root}
	Let $p$ be a prime and $n=p^k$. Then \[p=u(1-\zeta_n)^{\varphi(n)}\] where $u \in \ZZ[\zeta_n]^{\times}$.
\end{lemma}

\begin{lemma}\label{lemma:ideals_mult_to_power}
	Let $R$ be a Dedekind domain, $p$ a prime and $\gotha,\gothb,\gothc$ ideals such that \[\gotha\gothb=\gothc^p\] and suppose $\gotha,\gothb$ are coprime. Then there exist ideals $\mathfrak{e},\mathfrak{d}$ such that \[\gotha=\mathfrak{e}^p \qquad \gothb=\mathfrak{d}^p \qquad \mathfrak{e}\mathfrak{d}=\gothc\]
\end{lemma}


\section{Fermats Last Theorem for regular primes}

\begin{lemma}\label{lem:flt_fact_2}
	\lean{flt_fact_2}
	Let $p \geq 5$ be an prime number, $\zeta_p$ a $p$-th root of unity and $x, y \in \ZZ$ coprime.
	
	 For $i \neq j$ we can write \[(\zeta_p^i-\zeta_p^j)=u(1-\zeta_p)\] with $u$ a unit in $\ZZ[\zeta_p]$. From this it follows that the ideals \[(x+y),(x+\zeta_py),(x+\zeta_p^2y),\dots,(x+\zeta_p^{p-1}y)\] are pairwise coprime.
\end{lemma}

\begin{proof}
	 Lemma \ref{lemma:fac_of_p_in_p_th_root} gives that $u$ is a unit. So all that needs to be proved is that the ideals are coprime. Assume not, then for some $i \neq j$ we have some prime ideal $\gothp$ dividing by $(x+y\zeta_p^i)$ and $(x+y\zeta_p^j)$. It must then also divide their sum and their difference, so we must have $\gothp | (1-\zeta_p)$ or $\gothp | y$. Similarly, $\gothp$ divides $\zeta_p^j(x+y\zeta_p^i)-\zeta_p^i(x+y\zeta_p^j)$ so $\gothp$ divides $x$ or $(1-\zeta_p)$. We can't have $\gothp$ dividing $x,y$ since they are coprime, therefore $\gothp |(1-\zeta_p)$. We know that since $(1-\zeta_p)$ has norm $p$ it must be a prime ideal, so $\gothp=(1-\zeta_p)$. Now, note that $x+y \equiv x+y\zeta_+p^i \equiv 0 \mod \gothp$. But since $x,y \in \ZZ$ this means we would have $x+y \equiv 0 \pmod p$, which implies $z^p \equiv 0 \pmod p$ which contradicts our assumptions.
\end{proof}

\begin{lemma}\label{lem:flt_fact_3}
	\lean{flt_fact_3}
	
	 	Let $p$ be an prime number, $\zeta_p$ a $p$-th root of unity and $\a \in \ZZ[\zeta_p]$. Then $\a^p$ is congruent to an integer modulo $p$.
\end{lemma}

\begin{proof}
	  Just use $(x+y)^p \equiv x^p + y^p \pmod p$ and that $\zeta_p$ is a $p$-th root of unity.
\end{proof}

\begin{lemma}\label{lem:flt_fact_4}
	\lean{lem:flt_fact_4}
	Let $p \geq 5$ be an prime number, $\zeta_p$ a $p$-th root of unity and $ \a \in \ZZ[\zeta_p]$. If there is an integer $n$ such that $\a/n \in \ZZ[\zeta_p]$, then $n$ divides each $a_i$.
\end{lemma}

\begin{proof}
	 Looking at $\a=a_0+a_1\zeta_p+\cdots+a_{p-1}\zeta_p^{p-1}$, if one of the $a_i$'s is zero and $\a/n \in \ZZ[\zeta_p]$, then $\a/n=\sum_i a_i/n \zeta_p^i$. Now, as $\a/n \in \ZZ[\zeta_p]$, pick the basis of $\ZZ[\zeta_p]$ which does not contain $\zeta_p$ (which is possible as any subset of $\{1,\zeta_p,\dots,\zeta_p^{p-1}\}$ with $p-1$ elements forms a basis of $\ZZ[\zeta_p]$.]). Then $\a=\sum_i b_i \zeta_p^i$ where $b_i \in \ZZ$. Therefore comparing coefficients, we get the result.
\end{proof}

\begin{lemma}\label{lem:flt_fact_5}
	\lean{lem:flt_fact_5}
	\uses{lem:flt_fact_4,lem:flt_fact_3}
	Let $p \geq 5$ be an prime number, $\zeta_p$ a $p$-th root of unity and $ \a \in \ZZ[\zeta_p]$.  Suppose that $x+y\zeta_p^i=u \a^p$ with $u \in \ZZ[\zeta_p]^\times$ and $\a \in \ZZ[\zeta_p]$. Then there is an integer $k$ such that \[x+y\zeta_p^i-\zeta_p^{2k}x-\zeta_p^{2k-i}y \equiv 0 \pmod p.\]
\end{lemma}

\begin{proof}
 Using lemma \ref{lemma:unit_lemma} we have $(x+y\zeta_p^i) =\beta \zeta_p^k \a^p$ which is equivalent to $\beta \zeta_p^k a \pmod p$ with $a$ and integer \ref{lem:flt_fact_3}). Now, if we consider the complex conjugate we have $\ov{(x+y\zeta_p^i)  }\equiv \beta \zeta_p^{-k}a \pmod p$. Looking at $(x+y\zeta_p^i)-\zeta_p^{2k}\ov{(x+y\zeta_p^i)  }$ then gives the result.


\end{proof}


\begin{lemma}\label{lemma:may_assume}
	\uses{lemma:unit_lemma,_lemma:fac_of_p_in_p_th_root,lemma:ideals_mult_to_power,theorem:ring_of_ints_of_cyclo}

	Let $p \geq 5$ be an prime number, $\zeta_p$ a $p$-th root of unity and $K=\QQ(\zeta_p)$.  Assume that we have $x,y,z \in \ZZ$ with $\gcd(xyz,p)=1$ and such that \[x^p+y^p=z^p.\]

Then without loss of generality, we may assume $x,y,z$ are pairwise coprime and \[x \not \equiv y \mod p.\]
\end{lemma}



\begin{proof}
The first part is easy.

Reducing modulo $p$, using Fermat's little theorem, you get that if $x \equiv y \equiv -z \pmod p$ then $3z \equiv 0 \pmod p$. But since $p >3$ this means $p |z$ but this contradicts $\gcd(xyz,p)=1$. Now, if $x \equiv y \pmod p$ then  $x \not \equiv -z \pmod p$ we can relabel $y,z$ so that wlog $x \not \equiv y$ (this uses that $p$ is odd).

\end{proof}

\begin{definition}\label{defn:is_regular_number}
	\lean{is_regular_number}
	\leanok
	A prime number $p$ is called regular if it does not divide the class number of $\QQ(\zeta_p)$.
\end{definition}


\begin{theorem}\label{theorem:FLT_case_one}
	\uses{defn:is_regular_number, lem:flt_fact_5, lem:flt_fact_4, lemma:may_assume, lem:flt_fact_2}
	Let $p$ be an odd regular prime. Then \[x^p+y^p=z^p\] has no solutions with $x,y,z \in \ZZ$ and $\gcd(xyz,p)=1$.
\end{theorem}
\begin{proof}
	First thing is to note that if $x^p+y^p=z^p$ then \[z^p=(x+y)(x+\zeta_py)\cdots(x+y\zeta_p^{p-1})\] as ideals. Then since by \ref{lem:flt_fact_2} we know the ideals are coprime, then by lemma \ref{lemma:ideals_mult_to_power} we have that each $(x+y\zeta_p^i)=\gotha^p$, for $\gotha$ some ideal. Note that, $[\gotha^p]=1$ in the class group. Now, since $p$ does not divide the size of the class group we have that $[\gotha]=1$ in the class group, so its principal. So we have $x+y\zeta_p^i=u_i\a_i^p$ with $u_i$ a unit. So by \ref{lem:flt_fact_5} we have some $k$ such that $x+y\zeta_p-\zeta_p^{2k}x-\zeta_p^{2k-1} \equiv 0 \pmod p$. If $1,\zeta_p,\zeta_p^{2k},\zeta_p^{2k-1}$ are distinct, then \ref{lem:flt_fact_4} (which uses that $p>3$) says that $p$  divides $x,y$, contrary to our assumption. So they cannot be distinct, but checking each case leads to a contradiction, therefore there cannot be any such solutions.
\end{proof}

\begin{theorem}\label{theorem:FLT_case_two}\lean{flt_regular_case_two}
	\uses{lemma:may_assume, defn:is_regular_number}
	Let $p$ be an odd regular prime. Then \[x^p+y^p=z^p\] has no solutions with $x,y,z \in \ZZ$ and $p | xyz$.
\end{theorem}


\begin{theorem}\label{FLT_regular}
	\lean{flt_regular}
	\uses{theorem:FLT_case_one,theorem:FLT_case_two}
	Let $p$ be an odd regular prime.  Then \[x^p+y^p=z^p\] has no solutions with $x,y,z \in \ZZ$ and $xyz \ne 0$.
\end{theorem}

	
	
	\bibliographystyle{alpha}
	
	
	\bibliography{bibliog}
	
	
	
\end{document}	
